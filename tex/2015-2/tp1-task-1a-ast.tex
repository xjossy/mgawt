\documentclass[a4paper,10pt]{article}
\usepackage[margin=.3in]{geometry}
\usepackage[utf8]{inputenc}
\usepackage[russian]{babel}
\usepackage{makecell}


\usepackage{natbib}
\usepackage{amsmath}
\usepackage{amssymb}
\usepackage{enumitem}

\usepackage{wasysym}
\usepackage{multicol}
\usepackage{tabularx}
\usepackage{indentfirst}

\DeclareMathOperator{\xRe}{Re}
\DeclareMathOperator{\xIm}{Im}
\DeclareMathOperator{\xArg}{Arg}


\begin{document}
	\newbox\mgawtheader%
	\setbox\mgawtheader=\hbox{\small МГАВТ 2015 осень. Задание 1а* (в рамках допуска к экзамену) для группы ТП-1, \bf ВАР 1}%
	%
	\hbox{}

\def\inst{
	\vskip-\ht\mgawtheader\vskip-\parsep\vskip-.3cm\copy\mgawtheader%
	\vskip-\ht\mgawtheader\vskip-\parsep\vskip-.6cm\hbox{}%
	\section*{Задание 1а*. Переменные в формулах и аргументы}\vskip-.4cm
	\indent В логическую формулу каждая переменная может входить свободным и связанным образом. Переменная входит \textbf{свободным образом}, если она не стоит непосредственно после квантора $\forall$ или $\exists$ (не относится к квантору), либо находится вне формулы квантора, к которому относится. Остальные вхождения (непосредственно после квантора и внутри формулы квантора) называются \textbf{связанными} вхождениями. Переменные, входящие в формулу свободным образом, называются её \textbf{аргументами}. Логическая формула обретает значение (истину или ложь) только тогда, когда для всех аргументов заданы значения.

\textbf{Задача 1*} Обведите связанные вхождения переменных в формулу, подчеркните свободные вхождения переменных. Выпишите аргументы каждой из этих формул. \textit{Пример:}~$\exists\, \boxed{t}\; \varphi(\boxed{t},\underline{y}) \lor A(\underline{z})$, аргументы: $y$, $z$.\par\kern-.2cm
		\begin{multicols}{3}
			\begin{enumerate}[label=(\arabic*)\,]
				\item $\exists x\; \lnot\varphi(x,y)$;
				\item $\exists \zeta\forall y\; \psi(x,\zeta)\to\lnot\varphi(y,\zeta)$;
				\item $(\forall z\;\varphi(z,t))\land(\exists z\;\psi(z,\tau))$;
			\end{enumerate}
		\end{multicols}\par\kern-.2cm
Ту же операцию проделайте с формулами задачи 1.1.

\textbf{Задача 2*} Пусть $A(x,y)$~-- предикат <<$x=3y$>>, $O(x)$~-- <<$x$~-- чётное число>>. Проделать с формулой операции задачи 1*, определить, при каких аргументах формула истинна: а) $\forall s\; A(s,t) \to O(s)$; б) $\exists u\; A(y,u)$

\vskip\ht\mgawtheader\vskip\parsep\vskip.3cm\vskip.6cm
}
\inst\inst\inst\inst
\end{document}