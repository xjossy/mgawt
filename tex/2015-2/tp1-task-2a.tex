\documentclass[a4paper,10pt]{article}
\usepackage[margin=.3in]{geometry}
\usepackage[utf8]{inputenc}
\usepackage[russian]{babel}
\usepackage{makecell}


\usepackage{natbib}
\usepackage{amsmath}
\usepackage{amssymb}
\usepackage{enumitem}

\usepackage{wasysym}
\usepackage{multicol}
\usepackage{tabularx}
\usepackage{indentfirst}

\DeclareMathOperator{\xRe}{Re}
\DeclareMathOperator{\xIm}{Im}
\DeclareMathOperator{\xArg}{Arg}

\newcount\problemnum
\problemnum=1
\def\problem{\textbf{Задача \the\problemnum}\advance\problemnum by 1}

\begin{document}
	\newbox\mgawtheader%
	\setbox\mgawtheader=\hbox{\small МГАВТ 2015 осень. Задание 1в (в рамках допуска к экзамену) для группы ТП-1, \bf ВАР 1}%
	%
	\hbox{}
	\vskip-\ht\mgawtheader\vskip-\parsep\vskip-.3cm\copy\mgawtheader%
	\vskip-\ht\mgawtheader\vskip-\parsep\vskip-.6cm\hbox{}%
	\section*{Задание 2. Линейные пространства}\par\noindent

\textbf{Векторным (линейным) пространством} называется произвольное множество $V$ с введёнными на нём двумя операциями $+: V\times V\to V$ и $\cdot: \mathbb R\times V\to V$, которые называются сложением и умножением на числа и удовлетворяют ряду аксиом. Элементы векторного пространства называются векторами. Аксиомы, в частности, влекут существование особенного вектора $\vec 0$, удовлетворяющего соотношению $\forall v\in V\; v+\vec0=v$, и называемого \textbf{нулевым вектором}.

\textit{Пример:} $\mathbb R^n$, то есть множество всех записей вида $(x_1,\ldots,x_n)$, в которых в скобках записано через запятую $n$ произвольных действительных чисел. Сложение и умножение в $R^n$ определяются по формулам: $(x_1,\ldots,x_n)+(y_1,\ldots,y_n) = (x_1+y_1,\ldots,x_n+y_n)$, $a\cdot(x_1,\ldots,x_n) = (ax_1,\ldots,ax_n)$, где $x_i, y_i, a$~--- действительные числа.

\textbf{Линейной комбинацией} векторов $v_1$,\ldots,$v_n$ называется вектор, который можно получить из $v_1$,\ldots,$v_n$ применяя операции <<$+$>> и <<$\cdot$>>. Любую линейную комбинацию можно предствавить в виде $\alpha_1\cdot v_1+\cdots+\alpha_n\cdot v_n$, где $\alpha_1$, \ldots $\alpha_n$~--- некоторые действительные числа, называемые \textbf{коэффициентами} линейной комбинации. Векторы $v_1$,\ldots,$v_n$ называют \textbf{линейно независимыми}, если любая их линейная комбинация, в которой котя бы один коэффициент отличен от нуля, не равна нулевому вектору.

Отображения $f,g:A\to B$ являются равными, если их значения на каждом элементе $A$ совпадают, иными словами $f=g$ тогда и только тогда, когда $\forall x\in A\;f(x)=g(x)$.

Отображение может быть задано\par\kern-1.0em
\begin{itemize}
\setlength\itemsep{-.4em}
\item явно, перечислением образов элементов. Пример: $f:\{1,2\}\to\{100,200\}, 1\mapsto 200, 2\mapsto 100$;
\item формулой. Пример: $f:\mathbb R\to\mathbb R, f(x)=x^2$;
\end{itemize}
\par\kern-1.0em

\problem. Выпишите (задавая явно) все отображения между следующими конечными множествами $A$ и $B$ и определите их количество. Определите среди них инъективные, сюрьективные и биективные, а также их количество. \par\kern-1.0em

\begin{multicols}{3}
	\begin{enumerate}[label=(\arabic*)\,]
		\item $A=\{1,2\}$, $B=\{\sun,\bell\}$;
		\item $A=\{\sun,\bell,\smiley, \{1\}, \{1,2\}\}$, $B=\{\bell\}$;
		\item $A=\{\sun,\bell\}$, $B=A$;
		\item $A=\{\sun\}$,\\ $B=\{\bell,\sun,\smiley, \{1,4,5,6\}, \{1,2\}\}$;
		\item $A=\varnothing$, $B=\{\sun,\bell\}$;
		\item $A=\{\sun,\bell\}$, $B=\varnothing$;
		\item $A=\{1,2,3\}$, $B=\{\sun,\bell\}$;
		\item $A=\{1,2\}$, $B=\{\sun,\bell,100\}$;
		\item $A=\{1,2,3\}$, $B=A$;
	\end{enumerate}
\end{multicols}

\problem. Среди этих отображений найдите инъективные, сюрьективные, биективные: \par\kern-1.0em
\begin{multicols}{3}
	\begin{enumerate}[label=(\arabic*)\,]
		\item $f:\mathbb R\to\mathbb R, f(x)=x$;
		\item $f:\mathbb Z\to\mathbb Z, f(x)=-x$;
		\item $f:\mathbb N\to\mathbb Q, f(x)=x$;
		\item $f:\mathbb R\to\mathbb R, f(x)=x^2$;
		\item $f:\mathbb R\to\mathbb \{x\in\mathbb R|x\geqslant 0\}, f(x)=x^2$;
		\item $f:\mathbb N\to\mathbb N, f(x)=2x$;
		\item $f:\mathbb R\to\mathbb R, f(x)=2x$;
		\item $f:\mathbb Q\to\mathbb N$, $x$ переходит в знаменатель записи $x$ в виде несократимой дроби;
	\end{enumerate}
\end{multicols}

Если $f:A\to B$, а $g: B\to C$ для некоторых множеств $A$, $B$ и $C$, то определена операция \textit{композиции} $g\circ f: A\to C$, которая задана следующим образом: $g\circ f(x) = g(f(x))$.

\problem. Пусть $\mathrm{Country}$~--- множество стран, $\mathrm{City}$~--- множество городов. $f:\mathrm{Country}\to\mathrm{City}$ ставит в соответствие каждой стране столицу этой страны. $g:\mathrm{City}\to\mathrm{Country}$ ставит в соответствие каждому городу ту страну, в которой он находится. Определить, какие выражения корректны и вычислить их значения: \par\kern-1.0em
\begin{multicols}{3}
	\begin{enumerate}[label=(\arabic*)\,]
		\item $f(\text{Дрезден})$
		\item $g(\text{Новосибирск})$
		\item $f(\text{Россия})$
		\item $g(\text{Франция})$
		\item $f\circ g(\text{Колумбия})$
		\item $f\circ g(\text{Мумбай})$
		\item $g\circ f(\text{Солт Лейк Сити})$
		\item $g\circ f(\text{Венгрия})$
		\item $g\circ f\circ g(\text{Астана})$
	\end{enumerate}
\end{multicols}
\problem. Докажите, что композьция инъективных инъективна, композиция сюрьективных сюрьективна, композиция биективных биетивна. Если $g\circ f$ инъективно, верно ли, что а) $f$ инъективно? б) $g$ инъективно? Если $g\circ f$ сюрьективно, верно ли, что а) $f$ сюрьективно? б) $g$ сюрьективно?

Если $A$ множество, то отображение $id_A: A\to A$, задаваемое формулой $id_A(x)=x$ называется тождественным отображением множества $A$. Если $f: A\to B$, то отображение $g: B\to A$ называется обратным $f$, если $g\circ f=id_A$ и $f\circ g=id_B$.

\problem. а) Докажите, что обратные отображения имеются только у взаимно однозначных функций. б) Для каждого из биективных отображений задачи 1 найдите обратное. в) Выразите обратное композиции $f\circ g$ через обратные к $f$ и $g$.

Если существует взаимно однозначное отображение между множествами $A$ и $B$, то такие множества называются \textit{равномощными}, запись: $|A|=|B|$.

\problem. Докажите, что если $|A|=|B|$ и $|B|=|C|$, то $|A|=|C|$.

\problem. Докажите (находя взаимно однозначное отображение) равномощность следующих множеств: \par\kern-1.0em
\begin{multicols}{2}
	\begin{enumerate}[label=(\arabic*)\,]
		\item $\{1,2,3\}$ и $\{\sun,\smiley,\bell\}$;
		\item $\mathbb Z$ и $\mathbb Z$;
		\item $\mathbb Z$ и $\{x\in\mathbb Z|x>1\}$;
		\item $\mathbb Z$ и $2\mathbb Z=\{2x|x\in\mathbb Z\}$;
		\item $[0,1]$ и $[0,2]$ ($[0,1]$~--- это отрезок, то есть множество $\{x\in \mathbb R|0\leqslant x\leqslant 1\}$);
		\item $[0,1]$ и $[-1,1]$;
		\item $[0,1]$ и $[0,1)$ ($[0,1)$~--- это полуинтервал, то есть множество $\{x\in \mathbb R|0\leqslant x<1\}$);
		\item $[0,1)$ и $[0,\infty)$;
		\item $[0,1]$ и $\mathbb R$;
		\item любые два отрезка на плоскости;
		\item отрезок и окружность;
		\item множество всех подмножеств $\mathbb N$ и множество всех бесконечных последовательностей 0 и 1;
	\end{enumerate}
\end{multicols}

\problem (Теорема Кантора). Множество всех подмножеств множества $A$ обозначается через $2^A$. Рассматривая для отображения $f:A\to 2^A$ множество $\{x\in A|x\not\in f(x)\}$ докажите, что $A$ не может быть равномощно $2^A$.

\end{document}