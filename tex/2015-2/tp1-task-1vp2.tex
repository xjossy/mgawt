\documentclass[a4paper,10pt]{article}
\usepackage[margin=.3in]{geometry}
\usepackage[utf8]{inputenc}
\usepackage[russian]{babel}
\usepackage{makecell}


\usepackage{natbib}
\usepackage{amsmath}
\usepackage{amssymb}
\usepackage{enumitem}

\usepackage{wasysym}
\usepackage{multicol}
\usepackage{tabularx}
\usepackage{indentfirst}

\DeclareMathOperator{\xRe}{Re}
\DeclareMathOperator{\xIm}{Im}
\DeclareMathOperator{\xArg}{Arg}

\newcount\problemnum
\problemnum=1
\def\problem{\textbf{Задача \the\problemnum}\advance\problemnum by 1}

\begin{document}
\def\work{
	\newbox\mgawtheader%
	\setbox\mgawtheader=\hbox{\small МГАВТ 2015 осень. Задание 1в--продолжение (в рамках допуска к экзамену) для группы ТП-1, \bf ВАР 1}%
	%
	\hbox{}
	\vskip-\ht\mgawtheader\vskip-\parsep\vskip-.3cm\copy\mgawtheader%
	\vskip-\ht\mgawtheader\vskip-\parsep\vskip-.6cm\hbox{}%
	\section*{Задание 1в. Отображения (продолжение)}\par\noindent
\problemnum=9
\problem. Докажите (находя взаимно однозначное отображение) равномощность следующих множеств: \par\kern-1.0em
\begin{multicols}{2}
	\begin{enumerate}[label=(\arabic*)\,]
		\item $\{1,2,3\}$ и $\{\sun,\smiley,\bell\}$;
		\item $\mathbb Z$ и $\mathbb Z$;
		\item $\mathbb Z$ и $\{x\in\mathbb Z|x>1\}$;
		\item $\mathbb Z$ и $2\mathbb Z=\{2x|x\in\mathbb Z\}$;
		\item $[0,1]$ и $[0,2]$ ($[0,1]$~--- это отрезок, то есть множество $\{x\in \mathbb R|0\leqslant x\leqslant 1\}$);
		\item $[0,1]$ и $[-1,1]$;
		\item $[0,1]$ и $(0,1]$ ($(0,1]$~--- это полуинтервал, то есть множество $\{x\in \mathbb R|0< x\leqslant 1\}$);
		\item $(0,1]$ и $[1,\infty)$;
		\item $[0,1]$ и $\mathbb R$;
		\item любые два отрезка на плоскости;
		\item отрезок и окружность;
		\item множество всех подмножеств $\mathbb N$ и множество всех бесконечных последовательностей 0 и 1;
	\end{enumerate}
\end{multicols}
\par\kern-.5em
\problem\ (Теорема Кантора). Множество всех подмножеств множества $A$ обозначается через $2^A$. Рассматривая для отображения $f:A\to 2^A$ множество $\{x\in A|x\not\in f(x)\}$ докажите, что $A$ не может быть равномощно $2^A$.
\par\kern4em
}
\work\work\work

\end{document}