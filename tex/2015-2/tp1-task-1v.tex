\documentclass[a4paper,10pt]{article}
\usepackage[margin=.3in]{geometry}
\usepackage[utf8]{inputenc}
\usepackage[russian]{babel}
\usepackage{makecell}


\usepackage{natbib}
\usepackage{amsmath}
\usepackage{amssymb}
\usepackage{enumitem}

\usepackage{wasysym}
\usepackage{multicol}
\usepackage{tabularx}
\usepackage{indentfirst}

\DeclareMathOperator{\xRe}{Re}
\DeclareMathOperator{\xIm}{Im}
\DeclareMathOperator{\xArg}{Arg}


\begin{document}
	\newbox\mgawtheader%
	\setbox\mgawtheader=\hbox{\small МГАВТ 2015 осень. Задание 1в (в рамках допуска к экзамену) для группы ТП-1, \bf ВАР 1}%
	%
	\hbox{}
	\vskip-\ht\mgawtheader\vskip-\parsep\vskip-.3cm\copy\mgawtheader%
	\vskip-\ht\mgawtheader\vskip-\parsep\vskip-.6cm\hbox{}%
	\section*{Задание 1в. Отображения}\par\noindent

Если каждому элементу множества $A$ поставлен в соответствие ровно один элемент множества $B$, то говорят, что задана \textit{функция} (\textit{отображение}) на множестве $A$ со значениями в множестве $B$ (также: функция из $A$ в $B$). Обозначение $f: A\to B$ (читается: $f$ функция из $A$ в $B$). Элемент сопоставляемый элементу $x\in A$ называется образом $x$ при отображении $f$ и обозначается $f(x)$; также пишут $x\mapsto y$, если $y=f(x)$. Если $x\mapsto y$, то $y$ называется \textit{образом} $x$, а $x$ называется прообразом $y$.

Отображение $f:A\to B$ называется \textit{инъективным}, если у каждого $y\in B$ не более одного прообраза. Отображение $f:A\to B$ называется \textit{сюрьективным}, если у каждого $y\in B$ не менее одного прообраза. Отображение $f:A\to B$ называется \textit{биективным} или \textit{взаимно однозначным}, если у каждого $y\in B$ ровно один прообраз. 

Выпишите 

\end{document}